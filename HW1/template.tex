\documentclass[12pt]{article}
\usepackage[utf8]{inputenc}
\usepackage{float}
\usepackage{amsmath}


\usepackage[hmargin=3cm,vmargin=6.0cm]{geometry}
\topmargin=-2cm
\addtolength{\textheight}{6.5cm}
\addtolength{\textwidth}{2.0cm}
\setlength{\oddsidemargin}{0.0cm}
\setlength{\evensidemargin}{0.0cm}
\usepackage{indentfirst}
\usepackage{amsfonts}

\begin{document}

\section*{Student Information}

Name : \\

ID : \\


\section*{Answer 1}
\subsection*{a)}
The expected value is $E(x) = \sum P(x) x$. Therefore for the blue die, the proability is as shown below; \\ \\
\begin{center}
    \begin{tabular}{ c | c | c}
        $x$ & $P(x)$ & $P(x) \times x$ \\ 
        \hline
        $2$ & $4/6$ & $(2/4) \times 2$ \\
        $3$ & $1/6$ & $(1/6) \times 2$ \\
        $4$ & $1/6$ & $(1/6) \times 4$ \\
    \end{tabular}
\end{center}

\begin{equation}
    E_{blue}(x) = \sum P(x)\times x = 14/6
\end{equation}


The yellow die has a different expected value, and it is as follows;
\begin{center}
    \begin{tabular}{ c | c | c}
        $x$ & $P(x)$ & $P(x) \times x$ \\ 
        \hline
        $1$ & $2/6$ & $(2/4) \times 1$ \\
        $2$ & $2/6$ & $(2/6) \times 2$ \\
        $3$ & $2/6$ & $(2/6) \times 3$ \\
    \end{tabular}
\end{center}

\begin{equation}
    E_{yellow}(x) = \sum P(x)\times x = 12/6
\end{equation}



And lastly the expected value for the red die;
\begin{center}
    \begin{tabular}{ c | c | c}
        $x$ & $P(x)$ & $P(x) \times x$ \\ 
        \hline
        $1$ & $2/8$ & $(2/8) \times 1$ \\
        $2$ & $2/8$ & $(2/8) \times 2$ \\
        $3$ & $3/8$ & $(3/8) \times 3$ \\
        $5$ & $1/8$ & $(1/8) \times 5$ \\
    \end{tabular}
\end{center}

\begin{equation}
    E_{red}(x) = \sum P(x)\times x = 20/8
\end{equation}


\subsection*{b)}
\subsection*{c)}
\subsection*{d)}
\subsection*{e)}

\section*{Answer 2}
\subsection*{a)}
\subsection*{b)}
\subsection*{c)}
\subsection*{d)}
\subsection*{e)}
\subsection*{f)}

\end{document}
