\documentclass[12pt]{article}
\usepackage[utf8]{inputenc}
\usepackage{float}
\usepackage{amsmath}


\usepackage[hmargin=3cm,vmargin=6.0cm]{geometry}
\topmargin=-2cm
\addtolength{\textheight}{6.5cm}
\addtolength{\textwidth}{2.0cm}
\setlength{\oddsidemargin}{0.0cm}
\setlength{\evensidemargin}{0.0cm}
\usepackage{indentfirst}
\usepackage{amsfonts}

\begin{document}

\section*{Student Information}

Name : Atakan Onol \\

ID : 2102648 \\


\section*{Answer 1}
\subsection*{a)}
The expected value is $E(x) = \sum P(x) x$. Therefore for the blue die, the proability is as shown below; \\ \\
\begin{center}
    \begin{tabular}{ c | c | c}
        $x$ & $P(x)$ & $P(x) \times x$ \\ 
        \hline
        $2$ & $4/6$ & $(2/4) \times 2$ \\
        $3$ & $1/6$ & $(1/6) \times 2$ \\
        $4$ & $1/6$ & $(1/6) \times 4$ \\
    \end{tabular}
\end{center}

\begin{align*}
    E_{Blue}(x) = \sum P(x)\times x = 14/6
\end{align*}


The yellow die has a different expected value, and it is as follows;
\begin{center}
    \begin{tabular}{ c | c | c}
        $x$ & $P(x)$ & $P(x) \times x$ \\ 
        \hline
        $1$ & $2/6$ & $(2/4) \times 1$ \\
        $2$ & $2/6$ & $(2/6) \times 2$ \\
        $3$ & $2/6$ & $(2/6) \times 3$ \\
    \end{tabular}
\end{center}

\begin{align*}
    E_{Yellow}(x) = \sum P(x)\times x = 12/6
\end{align*}



And lastly the expected value for the red die;
\begin{center}
    \begin{tabular}{ c | c | c}
        $x$ & $P(x)$ & $P(x) \times x$ \\ 
        \hline
        $1$ & $2/8$ & $(2/8) \times 1$ \\
        $2$ & $2/8$ & $(2/8) \times 2$ \\
        $3$ & $3/8$ & $(3/8) \times 3$ \\
        $5$ & $1/8$ & $(1/8) \times 5$ \\
    \end{tabular}
\end{center}

\begin{align*}
    E_{Red}(x) = \sum P(x)\times x = 20/8
\end{align*}


\subsection*{b)}
We calculate the individual probabilities;
$P(2_{Red}) = 2/8$, $P(1_{Yellow}) = 2/6$, $P(2_{Yellow}) = 2/6$, and
$P(1_{Blue}) = 0$. We know that $E(x) = \sum xP(x)$, therefore we calculate 
expected value for each event.
\begin{align*}
    E(\text{2 red and 1 yellow}) = 2 \times 2/8 + 1 \times 2/6 = 5/6 \\
    E(\text{2 yellow and 1 blue}) = 2 \times 2/6 + 1 \times 0 = 4/6
\end{align*}

Therefore to maximize the total value, I would choose 2 red and 1 yellow.

\subsection*{c)}
We know that $P(2_{Yellow}) = 2/6$ and we can calculate $P(4_{Blue}) = 1/6$.
We then recalculate the expected value and compare.

\begin{align*}
    E(\text{2 yellow and 4 blue}) = 2 \times 2/6 + 4 \times 1/6 = 8/6
\end{align*}
 In this scenario, the expected value exceeds the expected value of 2 red and 
 1 yellow, therefore I would choose 2 yellow and 4 blue. 

\subsection*{d)}
The following probabilities are true; $P(3_{Red}) = 3/8$, $P(3_{Yellow}) = 2/6$,
and $P(3_{Blue}) = 1/6$. Since each probability of the event is independent, 
we can say that probability that a red die is rolled is equal to the probability that
a red 3 is rolled divided by the summation of the probability that a 3 is rolled for each die.

\begin{align*}
    P(\text{Die is red}) = \frac{1/3 \times 3/8}{1/3 \times (3/8 + 2/6 + 1/6)} = \frac{0.375}{0.875} \\
    P(\text{Die is red}) = 0.42857
\end{align*}

\subsection*{e)}
One can either roll a 3 with a red die and a 3 with the yellow die, or a 
5 with the red die and a 1 with the yellow die. Since the yellow die has 6
faces and the red die has 8 faces, there are a total of 48 different outcomes.
Since the red die has a value of 3 on 3 of its faces, and the yellow die
has the value of 3 on 2 of its faces, we multiply that for a total of 6 
combinations. The yellow die has a value of 1 on 2 of its faces and the red
has a value of 5 on 1 of its faces, if we multiply that we get a total of
2 different combinations. If we add the favorable outcomes and divide it 
by the total number of combinations, we get the value $P(x) = 8/48$.
\newline
\newline
Another way to approach this is, to calculate the proability to roll the favorable
values with each die, as such;

\begin{center}
    \begin{tabular}{ c | c }
        $x$ & $P(x)$  \\ 
        \hline
        $3_{Red}$       & $3/8$     \\
        $3_{Yellow}$    & $2/6$     \\
        $5_{Red}$       & $1/8$     \\
        $1_{Yellow}$    & $2/6$     \\
    \end{tabular}
\end{center}

Therefore the probability that a red 3 and a yellow 3 will be rolled its
$P(3_{Red},3_{Yellow}) = 3/8 \times 2/8 = 6/48$. And the probability to roll
a red 5 and a yellow 1 is $P(5_{Red}, 1_{Yellow}) = 1/8 \times 2/6 = 2/48$.
Since both events are independent, we can add the probabilities together so 
that $x$ is our favorable events, $P(x) = 6/48 + 2/48 = 8/48$.

\section*{Answer 2}
\subsection*{a)}
According to the provided table, there is only one event where $P(0,2) = 0.17$
\subsection*{b)}
The probability that there are two electric outages in Ankara is 0. Therefore
the number of electric outages in Istanbul is irrelavent because $P(2,0) = 0$
\subsection*{c)}
There are three different outcomes to satisfy the event that there will be 
exactly two outages in total. $P(0,2)$, $P(2,0)$, and $P(1,1)$. We already determined
in part b that $P(2,0) = 0$. Looking at the table, we know that $P(0,2) = 0.17$ and 
$P(1,1) = 0.11$. Therefore the probability that the given event will be satisfied
is $P(0,2) + P(2,0) + P(1,1) = 0.17 + 0 + 0.11 = 0.28$. So the probability is $0.28$.
\subsection*{d)}
The probability that Ankara will only have a single power outage is 
$\sum P(1,x)$ where $x$ is all real numbers. Therefore, the probability is 
$P(1,0) + P(1,1) + P(1,2) + P(1,3) = 0.12 + 0.11 + 0.22 + 0.15 = 0.60$ because
each event is mutually exclusive. 
\subsection*{e)}
To calculate the distibution, we use the cumulative distribution function.
To do so we add all of the probabilities together. 

\begin{align*}
    P(0,0) + P(0,1) + P(0,2) + P(0,3) + P(1,0) + P(1,1) + P(1,2) + P(1,3) \\
    = 0.08+0.13+0.17+0.02+0.12+0.11+0.22+0.15 = 1
\end{align*}

\subsection*{f)}
To be independent $P(A\cap I) = P(A)*P(I)$. Since we do not have the relavent data 
to prove it using this method, we have to deduce the following; We are told that this is 
a joint probability table and for joint probility to work, the events need to be independent.
Therefore we can safely say that the electric outages in Ankara and Istanbul are independent events.

\end{document}
